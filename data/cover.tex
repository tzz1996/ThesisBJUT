%封面与摘要中英文
\thusetup{
  udc={004}, %无需修改
  id={10005},%无需修改
  secretlevel={公开},%无需修改
  catalognumber={TP391.1}, %无需修改
  % ccovertype={硕\hspace{\fill}士\hspace{\fill}学\hspace{\fill}位\hspace{\fill}论\hspace{\fill}文},  % 彩色封面标题,学硕
  ccovertype={硕\hspace{\fill}士\hspace{\fill}专\hspace{\fill}业\hspace{\fill}学\hspace{\fill}位\hspace{\fill}论\hspace{\fill}文}, % 彩色封面标题,专硕
  % cthesistype={北京工业大学工学硕士学位论文}, % 内封面标题,学硕
  cthesistype={北京工业大学硕士专业学位论文}, % 内封面标题,专硕
  %cthesistypep={(非全日制)}, % 内封面标题,全日制或非全日制或同等学力等。学硕请忽略。
  % cpageheading={北京工业大学工学硕士学位论文}, % 页眉,学硕
  cpageheading={北京工业大学工程硕士专业学位论文}, % 页眉,专硕
  cstudent={S201861807},
  ctitle={基于UEFI的硬盘文件安全加载策略设计与实现},
  cauthor={唐治中},
  cdepartment={计算机技术},
  cmajor={信息安全},
  cdegree={工程硕士专业学位},
  csupervisor={张建标\ \ 教授},
  ccollege={信息学部计算机学院},  
  cdate={2021年6月},
  corganization={北京工业大学},
  %
  %=========
  % 英文信息
  %=========
  % ecovertype={MASTERAL\  DISSERTATION}, % 学硕
  ecovertype={PROFESSIONAL\  master\  DISSERTATION}, % 专硕
  etitle={Design and Implementation of Hard Disk File Security Loading Strategy Based on UEFI},
  edegree={Master of Engineering},
  emajor={Computer Science and Technology},
  eauthor={Tang Zhizhong},
  esupervisor={Associate Professor Wei Ma}
}

% 定义中英文摘要和关键字
% 摘要中文大致一页长度 一段背景引入本文研究 说现有问题总结 本文贡献主要创新点 3个研究内容 3个创新点
\begin{cabstract}

随着科学的发展,技术的进步,越来越多的计算机上层应用得到极大的发展,例如人工智能和区块链、物联网等,而这些
技术的发展和流行,都要依赖于操作系统或更底层系统的功能支持和安全保障。《信息安全技术 网络安全等级保护基本要求》
国家标准中就指出,实际运行中的关键系统需要进行各阶段的可信验证工作。UEFI规范作为近些年来替代传统BIOS的一种
底层系统规范,从各个机构的服务器到每个人的个人计算机,都得到了广泛的应用。UEFI BIOS在可扩展性和开发效率、开发
难易程度上都得到极大的增强,但正是这种C语言来替代传统汇编语言的方案,使得UEFI BIOS同样会遭受C语言代码的攻击,
不同处理器架构的统一规范开发方式也为攻击底层固件系统提供了更多的条件。从硬盘这样的块设备经过UEFI BIOS启动操作
系统仍然是如今主流的系统启动方式,他涉及到的不光是硬盘设备的被攻击的可能性,也涉及到固件系统层面的攻击,如对于
UEFI文件系统这样的操作硬盘的内部逻辑,就更易收到针对性的攻击对硬盘文件的安全性带来威胁。因此研究基于UEFI的硬盘
文件的安全加载就具有重要意义。本文主要工作如下:
\par (1)针对UEFI BIOS系统加载硬盘文件过程中存在的安全威胁,通过研究硬盘设备用于存放UEFI BIOS环境中可访问数据的
ESP分区的组织结构,研究UEFI BIOS环境中读取硬盘文件的方式,以及研究以UEFI型驱动程序作为基础的可信度量机制,以此
确保针对硬盘设备攻击的文件安全性。结合可信计算理论提出基于UEFI的硬盘设备文件可信加载的总体框架,研究解决UEFI BIOS
环境加载硬盘文件过程中的安全可信。
\par (2)针对UEFI BIOS系统的启动阶段设计,通过研究各个阶段的加载功能和特点,结合可信计算技术,确定每个阶段
需要度量的固件芯片中的代码内容,以UEFI启动的第一阶段作为整个系统的可信根,逐个度量后面阶段的核心代码,将信任链
逐阶段传递,一直到UEFI加载硬盘文件的BDS阶段,以保证加载文件前,系统的安全可信性。
\par (3)针对系统中使用的可信计算平台,设计出以BMC系统作为可信平台的方案,通过特定的BMC通信协议和接口对其
进行通信功能的开发,并根据UEFI启动阶段的特性,设计出PEI阶段和DXE阶段分别开发BMC驱动程序的方案,来达到各个
阶段度量特定驱动和代码的效果。
\par 

\end{cabstract}

\ckeywords{UEFI, 文件系统协议栈, BMC, 可信计算, 信任链}

\begin{eabstract}

With the development of science and technology, more and more computer upper-level applications 
have been greatly developed, such as artificial intelligence, blockchain, and the Internet of Things. 
The development and popularity of these technologies depend on operating systems. Or the functional 
support and security guarantee of the lower-level system. According to the national standard of 
"Information Security Technology Network Security Graded Protection Basic Requirements", critical 
systems in actual operation need to be trusted at various stages. As a low-level system specification 
that replaces traditional BIOS in recent years, UEFI specification has been widely used from servers 
of various organizations to personal computers of everyone. UEFI BIOS has been greatly enhanced in 
terms of scalability, development efficiency, and development difficulty. However, it is this C 
language that replaces the traditional assembly language program, which makes UEFI BIOS also subject 
to C language code attacks. The unified specification development method of the processor 
architecture also provides more conditions for attacking the underlying firmware system. 
Starting the operating system from a block device such as a hard disk through UEFI BIOS is 
still the mainstream system startup method today. It involves not only the possibility of the 
hard disk device being attacked, but also the firmware system level attack, such as the UEFI 
file system. Such internal logic of operating the hard disk makes it easier to receive 
targeted attacks that threaten the security of hard disk files. Therefore, it is of great 
significance to study the secure loading of UEFI-based hard disk files. The main work of this 
paper is as follows:
\par (1)In view of the security threats in the process of loading hard disk files in the UEFI 
BIOS system, the organization structure of the ESP partition of the hard disk device used to store 
the accessible data in the UEFI BIOS environment is studied, the method of reading hard disk files 
in the UEFI BIOS environment, and the research UEFI-type drivers are used as the basic credibility 
measurement mechanism to ensure file security against hard disk device attacks. Combining with 
the theory of trusted computing, a general framework for trusted loading of hard disk device files 
based on UEFI is proposed, and the security and trustworthiness of the process of loading hard 
disk files in the UEFI BIOS environment is studied.
\par (2)Aiming at the design of the boot phase of the UEFI BIOS system, by studying the loading 
functions and characteristics of each phase, combined with trusted computing technology, determine 
the code content in the firmware chip that needs to be measured at each phase, and the first phase 
of UEFI boot as the entire system The root of trust measures the core code of the later stages 
one by one, and passes the chain of trust stage by stage until the BDS stage of UEFI loading hard 
disk files to ensure the security and credibility of the system before loading files.
\par (3)Aiming at the trusted computing platform used in the system, the BMC system is designed 
as a trusted platform, and the BMC communication protocol and interface
Carry out the development of communication functions, and according to the characteristics of the 
UEFI boot phase, design the PEI phase and the DXE phase to develop the BMC driver program separately 
to achieve each Phase measures the effect of specific drivers and codes.

\end{eabstract}

\ekeywords{UEFI, File system protocol stack, BMC, trusted computing, Chain of trust}
