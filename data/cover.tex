%封面与摘要中英文
\thusetup{
  udc={004}, %无需修改
  id={10005},%无需修改
  secretlevel={公开},%无需修改
  catalognumber={TP391.1}, %无需修改
  % ccovertype={硕\hspace{\fill}士\hspace{\fill}学\hspace{\fill}位\hspace{\fill}论\hspace{\fill}文},  % 彩色封面标题,学硕
  ccovertype={硕\hspace{\fill}士\hspace{\fill}专\hspace{\fill}业\hspace{\fill}学\hspace{\fill}位\hspace{\fill}论\hspace{\fill}文}, % 彩色封面标题,专硕
  % cthesistype={北京工业大学工学硕士学位论文}, % 内封面标题,学硕
  cthesistype={北京工业大学硕士专业学位论文}, % 内封面标题,专硕
  %cthesistypep={(非全日制)}, % 内封面标题,全日制或非全日制或同等学力等。学硕请忽略。
  % cpageheading={北京工业大学工学硕士学位论文}, % 页眉,学硕
  cpageheading={北京工业大学工程硕士专业学位论文}, % 页眉,专硕
  cstudent={S201861807},
  ctitle={基于UEFI的硬盘文件安全加载系统设计与实现},
  cauthor={唐治中},
  cdepartment={计算机技术},
  cmajor={信息安全},
  cdegree={工程硕士专业学位},
  csupervisor={张建标\ \ 教授},
  ccollege={信息学部计算机学院},  
  cdate={2021年6月},
  corganization={北京工业大学},
  %
  %=========
  % 英文信息
  %=========
  % ecovertype={MASTERAL\  DISSERTATION}, % 学硕
  ecovertype={PROFESSIONAL\  master\  DISSERTATION}, % 专硕
  etitle={Design and Implementation of Hard Disk File Security Loading Strategy Based on UEFI},
  edegree={Master of Engineering},
  emajor={Computer Science and Technology},
  eauthor={Tang Zhizhong},
  esupervisor={Associate Professor Wei Ma}
}

% 定义中英文摘要和关键字
% 摘要中文大致一页长度 一段背景引入本文研究 说现有问题总结 本文贡献主要创新点 3个研究内容 3个创新点
\begin{cabstract}

% 随着科学的发展,技术的进步,越来越多的计算机上层应用得到极大的发展,例如人工智能和区块链、物联网等,而这些
% 技术的发展和流行,都要依赖于操作系统或更底层系统的功能支持和安全保障。
在一个计算机系统中,底层软件系统占有着重要的位置,他不仅为上层系统
和应用提供基础接口,也从底层为整个系统的安全和可信提供了保障。《信息安
全技术网络安全等级保护基本要求》国家标准中就指出,实际运行中的关键系统
需要进行各阶段的可信验证工作。UEFI 规范作为近些年来替代传统 BIOS 的一
种底层系统规范,从各个机构的服务器到每个人的个人计算机,都得到了广泛的
应用。从硬盘这样的块设备经过 UEFI BIOS 启动操作系统仍然是如今主流的系
统启动方式,所涉及到的不光是硬盘设备的被攻击的可能性,也涉及到固件系统
层面的攻击,如对于 UEFI 文件系统这样的操作硬盘的内部逻辑,就更易受到针
对性的攻击,对硬盘文件的安全性带来威胁。
\par 在 UEFI 环境中加载硬盘文件需要经过 UEFI 文件系统协议栈,而协议栈中
的协议函数则依赖于 UEFI 启动阶段所加载的文件系统环境对应的驱动程序,确
定需要度量的文件系统协议栈相关驱动程序也成为了本文的研究关键。而 BIOS
的驱动文件的存储位置位于闪存设备中,通过固件文件系统格式进行存储,要
在 UEFI 中对驱动进行度量和加载就需要解析驱动存储格式并对关键信息进行提
取。因此研究基于 UEFI 的硬盘文件的安全加载就具有重要意义。本文主要工作
如下:
% \par (1)针对UEFI BIOS系统加载硬盘文件过程中存在的安全威胁,通过研究硬盘设备用于存放UEFI BIOS环境中可访问数据的
% ESP分区的组织结构,研究UEFI BIOS环境中读取硬盘文件的方式,以及研究以UEFI型驱动程序作为基础的可信度量机制,以此
% 确保针对硬盘设备攻击的文件安全性。结合可信计算理论提出基于UEFI的硬盘设备文件可信加载的总体框架,研究解决UEFI BIOS
% 环境加载硬盘文件过程中的安全可信。
% \par (2)针对UEFI BIOS系统的启动阶段设计,通过研究各个阶段的加载功能和特点,结合可信计算技术,确定每个阶段
% 需要度量的固件芯片中的代码内容,以UEFI启动的第一阶段作为整个系统的可信根,逐个度量后面阶段的核心代码,将信任链
% 逐阶段传递,一直到UEFI加载硬盘文件的BDS阶段,以保证加载文件前,系统的安全可信性。
% \par (3)针对系统中使用的可信计算平台,设计出以BMC系统作为可信平台的方案,通过特定的BMC通信协议和接口对其
% 进行通信功能的开发,并根据UEFI启动阶段的特性,设计出PEI阶段和DXE阶段分别开发BMC驱动程序的方案,来达到各个
% 阶段度量特定驱动和代码的效果。
% \par 

\par (1)针对UEFI BIOS系统加载硬盘文件过程中存在的安全威胁,结合可信计算理论提出基于UEFI的硬盘设备文件可信
加载系统的总体框架,并使用的底层可信平台BMC及基板管理控制器,实现UEFI BIOS中BMC驱动程序。
\par (2)针对UEFI BIOS系统加载UEFI BIOS存储闪存设备中的驱动文件过程中存在的安全问题,设计提出UEFI BIOS启动
阶段所需度量的UEFI文件系统协议栈驱动程序,并实现度量模块功能。
\par (3)针对安全方案中的可信度量功能,设计并实现了通过BMC的日志存储功能,并完成确保驱动按顺序加载的依赖表达式
编写。
\par (4)根据本系统安全方案的设计,在申威平台中对驱动度量模块、日志生成功能和驱动加载
顺序修改功能进行实现和测试,以保证功能开发过程的有效性和安全方案的可实施性。

\end{cabstract}

\ckeywords{UEFI, 文件系统协议栈, 可信计算, 信任链}

\begin{eabstract}
% With the development of science and technology, more and more computer upper-level applications 
% have been greatly developed, such as artificial intelligence, blockchain, and the Internet of Things. 
% The development and popularity of these technologies depend on operating systems. Or the functional 
% support and security guarantee of the lower-level system. 
In a computer system, the underlying software system occupies an important posi-
tion. Itnotonlyprovidesabasicinterfacefortheuppersystemandapplications, butalso
provides a guarantee for the safety and credibility of the entire system from the bottom.
According to the national standard of "Information Security Technology Network Secu-
rity Graded Protection Basic Requirements", critical systems in actual operation need to
be trusted at various stages. As a low-level system specification that replaces traditional
BIOS in recent years, UEFI specification has been widely used from servers of various
organizations to personal computers of everyone. Starting the operating system from
a block device such as a hard disk through UEFI BIOS is still the mainstream system
startup method today. It involves not only the possibility of the hard disk device being
attacked, but also the firmware system level attack, such as the UEFI file system. Such
internal logic of operating the hard disk makes it easier to receive targeted attacks that
threaten the security of hard disk files. Therefore, it is of great significance to study the
secure loading of UEFI-based hard disk files.
\par Loading a hard disk file in the UEFI environment needs to go through the UEFI
file system protocol stack, and the protocol functions in the protocol stack depend on the
driver corresponding to the file system environment loaded in the UEFI startup phase,
and determine the file system protocol stack related drivers that need to be measured It
has also become the key to the study of this article. The storage location of the BIOS
driver file is located in the flash memory device and is stored in the firmware file system
format. To measure and load the driver in UEFI, it is necessary to parse the driver
storage format and extract key information. Therefore, it is of great significance to
study the secure loading of hard disk files based on UEFI. The main work of this paper
is as follows:
\par (1)Aiming at the security threats in the process of loading hard disk files in the UEFI BIOS 
system, combined with trusted computing theory, a UEFI-based overall framework for trusted loading 
of hard disk device files is proposed, and the underlying trusted platform BMC and baseboard 
management controller are used to implement UEFI BIOS In the BMC driver.
\par (2)Aiming at the security problems in the process of loading the driver files in the UEFI BIOS 
storage flash memory device in the UEFI BIOS system, the UEFI file system protocol stack driver 
program for the measurement required by the UEFI BIOS startup phase is designed and proposed, and 
the measurement module function is realized.
\par (3)Aiming at the credibility measurement function in the security scheme, the log storage 
function through BMC is designed and realized, and the dependency expression writing to ensure that 
the driver is loaded in order.
\par (4)According to the design of the safety scheme of this system, the drive measurement module, 
log generation function and drive loading sequence modification function are implemented and 
tested in the Sunway platform to ensure the effectiveness of the function development process 
and the implementability of the safety scheme.

\end{eabstract}

\ekeywords{UEFI, File system protocol stack, trusted computing, Chain of trust}
