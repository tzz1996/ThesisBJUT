%封面与摘要中英文
\thusetup{
  udc={004}, %无需修改
  id={10005},%无需修改
  secretlevel={公开},%无需修改
  catalognumber={TP391.1}, %无需修改
  % ccovertype={硕\hspace{\fill}士\hspace{\fill}学\hspace{\fill}位\hspace{\fill}论\hspace{\fill}文},  % 彩色封面标题,学硕
  ccovertype={硕\hspace{\fill}士\hspace{\fill}专\hspace{\fill}业\hspace{\fill}学\hspace{\fill}位\hspace{\fill}论\hspace{\fill}文}, % 彩色封面标题,专硕
  % cthesistype={北京工业大学工学硕士学位论文}, % 内封面标题,学硕
  cthesistype={北京工业大学硕士专业学位论文}, % 内封面标题,专硕
  %cthesistypep={(非全日制)}, % 内封面标题,全日制或非全日制或同等学力等。学硕请忽略。
  % cpageheading={北京工业大学工学硕士学位论文}, % 页眉,学硕
  cpageheading={北京工业大学工程硕士专业学位论文}, % 页眉,专硕
  cstudent={S201861807},
  ctitle={基于UEFI的硬盘文件安全加载策略研究与实现},
  cauthor={唐治中},
  cdepartment={计算机技术},
  cmajor={信息安全},
  cdegree={工程硕士专业学位},
  csupervisor={张建标\ \ 教授},
  ccollege={信息学部计算机学院},  
  cdate={2020年12月},
  corganization={北京工业大学},
  %
  %=========
  % 英文信息
  %=========
  % ecovertype={MASTERAL\  DISSERTATION}, % 学硕
  ecovertype={PROFESSIONAL\  master\  DISSERTATION}, % 专硕
  etitle={Research on Multi-view Complementarity based SLAM},
  edegree={Master of Engineering},
  emajor={Computer Science and Technology},
  eauthor={Tang Zhizhong},
  esupervisor={Associate Professor Wei Ma}
}

% 定义中英文摘要和关键字
% 摘要中文大致一页长度 一段背景引入本文研究 说现有问题总结 本文贡献主要创新点 3个研究内容 3个创新点
\begin{cabstract}

随着科学的发展,技术的进步,越来越多的计算机上层应用得到极大的发展,例如人工智能和区块链、物联网等,而这些
技术的发展和流行,都要依赖于操作系统或更底层系统的功能支持和安全保障。《信息安全技术 网络安全等级保护基本要求》
国家标准中就指出,实际运行中的关键系统需要进行各阶段的可信验证工作。UEFI规范作为近些年来替代传统BIOS的一种
底层系统规范,从各个机构的服务器到每个人的个人计算机,都得到了广泛的应用。UEFI BIOS在可扩展性和开发效率、开发
难易程度上都得到极大的增强,但正是这种C语言来替代传统汇编语言的方案,使得UEFI BIOS同样会遭受C语言代码的攻击,
不同处理器架构的统一规范开发方式也为攻击底层固件系统提供了更多的条件。从硬盘这样的块设备经过UEFI BIOS启动操作
系统仍然是如今主流的系统启动方式,他涉及到的不光是硬盘设备的被攻击的可能性,也涉及到固件系统层面的攻击,如对于
UEFI文件系统这样的操作硬盘的内部逻辑,就更易收到针对性的攻击对硬盘文件的安全性带来威胁。因此研究基于UEFI的硬盘
文件的安全加载就具有重要意义。本文主要工作如下:
\par (1)针对UEFI BIOS系统加载硬盘文件过程中存在的安全威胁,通过研究硬盘设备用于存放UEFI BIOS环境中可访问数据的
ESP分区的组织结构,研究UEFI BIOS环境中读取硬盘文件的方式,以及研究以UEFI型驱动程序作为基础的可信度量机制,以此
确保针对硬盘设备攻击的文件安全性。结合可信计算理论提出基于UEFI的硬盘设备文件可信加载的总体框架,研究解决UEFI BIOS
环境加载硬盘文件过程中的安全可信。
\par (2)针对UEFI BIOS系统的启动阶段设计,通过研究各个阶段的加载功能和特点,结合可信计算技术,确定每个阶段
需要度量的固件芯片中的代码内容,以UEFI启动的第一阶段作为整个系统的可信根,逐个度量后面阶段的核心代码,将信任链
逐阶段传递,一直到UEFI加载硬盘文件的BDS阶段,以保证加载文件前,系统的安全可信性。
\par (3)针对系统中使用的可信计算平台,设计出以BMC系统作为可信平台的方案,通过特定的BMC通信协议和接口对其
进行通信功能的开发,并根据UEFI启动阶段的特性,设计出PEI阶段和DXE阶段分别开发BMC驱动程序的方案,来达到各个
阶段度量特定驱动和代码的效果。
\par 

\end{cabstract}

\ckeywords{UEFI, 文件系统协议栈, BMC, 可信计算, 信任链}

\begin{eabstract}

English abstract

\end{eabstract}

\ekeywords{simultaneous localization and mapping, multi-view complementarity, bundle adjustment, line sequence optimization, complex feature optimization}
