%封面与摘要中英文
\thusetup{
  udc={004}, %无需修改
  id={10005},%无需修改
  secretlevel={公开},%无需修改
  catalognumber={TP391.1}, %无需修改
  % ccovertype={硕\hspace{\fill}士\hspace{\fill}学\hspace{\fill}位\hspace{\fill}论\hspace{\fill}文},  % 彩色封面标题,学硕
  ccovertype={硕\hspace{\fill}士\hspace{\fill}专\hspace{\fill}业\hspace{\fill}学\hspace{\fill}位\hspace{\fill}论\hspace{\fill}文}, % 彩色封面标题,专硕
  % cthesistype={北京工业大学工学硕士学位论文}, % 内封面标题,学硕
  cthesistype={北京工业大学硕士专业学位论文}, % 内封面标题,专硕
  %cthesistypep={(非全日制)}, % 内封面标题,全日制或非全日制或同等学力等。学硕请忽略。
  % cpageheading={北京工业大学工学硕士学位论文}, % 页眉,学硕
  cpageheading={北京工业大学工程硕士专业学位论文}, % 页眉,专硕
  cstudent={S201861807},
  ctitle={基于UEFI的硬盘文件安全加载策略研究与实现},
  cauthor={唐治中},
  cdepartment={计算机技术},
  cmajor={信息安全},
  cdegree={工程硕士专业学位},
  csupervisor={张建标\ \ 教授},
  ccollege={信息学部计算机学院},  
  cdate={2020年12月},
  corganization={北京工业大学},
  %
  %=========
  % 英文信息
  %=========
  % ecovertype={MASTERAL\  DISSERTATION}, % 学硕
  ecovertype={PROFESSIONAL\  master\  DISSERTATION}, % 专硕
  etitle={Research on Multi-view Complementarity based SLAM},
  edegree={Master of Engineering},
  emajor={Computer Science and Technology},
  eauthor={Tang Zhizhong},
  esupervisor={Associate Professor Wei Ma}
}

% 定义中英文摘要和关键字
% 摘要中文大致一页长度 一段背景引入本文研究 说现有问题总结 本文贡献主要创新点 3个研究内容 3个创新点
\begin{cabstract}

中文摘要内容

\end{cabstract}

\ckeywords{同步定位与地图构建, 多视角互补, 捆绑优化, 线条优化, 复合特征优化}

\begin{eabstract}

English abstract

\end{eabstract}

\ekeywords{simultaneous localization and mapping, multi-view complementarity, bundle adjustment, line sequence optimization, complex feature optimization}
